The challenge of quantifying uncertainty propagation in real-world systems is 
rooted in the high-dimensionality of the stochastic input and the frequent lack 
of explicit knowledge of its probability distribution. Traditional approaches 
show limitations for such problems, especially when the size of the training data is limited.
To address these difficulties, we have developed a general framework of constructing 
surrogate models on spaces of stochastic input with arbitrary probability measure 
irrespective of \blue{the mutual dependencies between individual components
of the random inputs} and the analytical form.
The present \acf*{DSRAR} framework 
includes a \blue{data-driven construction of multivariate polynomial basis for arbitrary mutually
dependent probability measures} and a sparsity enhancement rotation procedure.
This sparsity-enhancing rotation method was initially proposed in our previous work \cite{Lei_Yang_MMS_2015} 
for Gaussian density distributions, which may not be feasible for non-Gaussian distributions due 
to the loss of orthogonality after the rotation. To remedy such difficulties, \blue{we 
developed a new data-driven approach to construct orthonormal polynomials for arbitrary mutually dependent randomness,
ensuring the constructed basis} maintains the orthogonality/near-orthogonality 
with respect to the density of the rotated random vector, \blue{where 
directly applying the regular polynomial chaos including arbitrary polynomial chaos (aPC) \cite{OLADYSHKIN2012} 
shows limitations due to the assumption of the mutual independence between 
the components of the random inputs.}   
The developed \acs*{DSRAR} framework leads to accurate recovery, with only limited training data, of a sparse representation 
of the target functions. 
The effectiveness of our method is demonstrated in challenging problems such as \aclp*{PDE} and realistic molecular systems 
\textcolor{blue}{within high-dimensional ($O(10)$) conformational spaces} where the underlying density is implicitly represented by a \textcolor{blue}{large collection of sample data}, as well as systems {with explicitly given} non-Gaussian probabilistic measures.

