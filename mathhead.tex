%\documentclass[a4paper,11pt]{amsart}
%\documentclass[a4paper,11pt]{article}
\usepackage{amsmath}
\usepackage{amsfonts}
\usepackage{amssymb}
\usepackage{amsthm}
\usepackage{bm}
\usepackage{indentfirst}
%\usepackage{titlesec}
\usepackage{graphicx}
\usepackage{subfigure}
\usepackage{array}
\usepackage{fullpage}
\usepackage[hidelinks]{hyperref}
\usepackage{color}
\usepackage{latexsym}
\usepackage{soul}
\usepackage{etoolbox}
\usepackage{acronym}
\usepackage[super]{nth}
\usepackage[draft,inline,nomargin]{fixme}

\mathchardef\mhyphen="2D
\newcommand{\angstrom}{\mbox{\normalfont\AA}}
\newtheorem{Definition}{definition}

%\renewcommand{\baselinestretch}{1.0}

\newcommand{\ignore}[1]{}

%\def\mb{\mathbf}
\def\half{\frac{1}{2}}
\def\veps{\varepsilon}
\def\dsp{\displaystyle}
\def\cal{\mathcal}
%\def\bm{\boldsymbol}
\newcommand{\Rmnum}[1]{\uppercase\expandafter{\romannumeral #1\relax}}
%\usepackage[top=1.2in, bottom=1.3in, left=0.9in, right=0.9in]{geometry}

\newcommand{\D}{\displaystyle}
\newcommand{\mb}{\mathbf}
\newcommand{\mo}{\mathcal{O}}
\DeclareMathAlphabet{\mathsfsl}{OT1}{cmss}{m}{sl}
\newcommand{\tensormat}[1]{\mathsfsl{#1}}
\renewcommand{\vec}[1]{\mathit{\boldsymbol{#1}}}
\newcommand{\dif}{\,\mathrm{d}}
\newcommand{\mi}{\mathrm{i}}
\newcommand{\me}{\mathrm{e}}
\newcommand{\ba}{\bm\alpha}
\newcommand{\bx}{\bm\xi}
\newcommand{\ve}{\varepsilon}
\newcommand{\PreserveBackslash}[1]{\let\temp=\\#1\let\\=\temp}
\newcolumntype{C}[1]{>{\PreserveBackslash\centering}p{#1}}
\newcolumntype{R}[1]{>{\PreserveBackslash\raggedleft}p{#1}}
\newcolumntype{L}[1]{>{\PreserveBackslash\raggedright}p{#1}}

%\newcommand{\prb}[1]{\noindent\textbf{Problem #1 :}}
%\newcommand{\prf}{\noindent\textbf{Proof :}}
%\newcommand{\sln}{\noindent\textbf{Solution :}}

\renewcommand{\theequation}{\thesection.\arabic{equation}}
\numberwithin{equation}{section}
\newtheorem{thm}{Theorem}[section]
\newtheorem{lem}[thm]{Lemma}
\newtheorem{cor}{Corollary}
\theoremstyle{definition}
\newtheorem{defn}[thm]{Definition}
\newtheorem{prop}[thm]{Property}
\newtheorem{rem}[thm]{Remark}
\newtheorem{exam}{Example}
\newcommand\Def{\stackrel{\textrm{def}}{=}}
%
%\renewcommand{\baselinestretch}{1.40}
\renewcommand{\thefootnote}{\fnsymbol{footnote}}

\usepackage{newfloat, algcompatible}
\usepackage{algcompatible}
\DeclareFloatingEnvironment[fileext=loa, listname=List of Algorithms, name=ALGORITHM, placement=tbhp]{algorithm}

\acrodef{QoI}{quantity of interest}
\acrodef{UQ}{uncertainty quantification}
\acrodef{MC}{Monte Carlo}
\acrodef{GP}{Gaussian Process}
\acrodef{gPC}{generalized polynomial chaos}
\acrodef{aPC}{arbitrary polynomial chaos}
\acrodef{amdP}{\emph{polynomials for arbitrary mutually dependent}}
\acrodef{DSRAR}{\emph{Data-driven Sparsity-enhancing Rotation for Arbitrary Randomness}} 
\acrodef{CS}{compressed sensing}
\acrodef{SASA}{solvent-accessible surface area}
\acrodef{PDE}{partial differential equation}
\acrodef{RIC}{restricted isometry constant}
\acrodef{RIP}{restricted isometry property}
\acrodef{PCA}{principal component analysis}
\acrodef{MD}{molecular dynamics}
\acrodef{APBS}{Adaptive Poisson-Boltzmann Solver}
\acrodef{PDF}{probability density function}
\acrodef{GAFF}{General AMBER Force Field}


\usepackage{tikz}
\usetikzlibrary{patterns}

\newcommand{\dashline}{\raisebox{0pt}{\tikz{\draw[-,dashed,line width = 1.0pt](0.,1mm) -- (10.0mm,1mm)}}}

\newcommand{\rectangleopen}{\raisebox{0pt}{\tikz{\draw[solid,line width = 1.0pt](4.mm,0)rectangle (5.8mm,1.8mm);
  }}}

\newcommand{\circleopen}{\raisebox{0pt}{\tikz{\draw[solid, line width = 1.0pt](5mm,0) circle[radius=1.0mm];
}}}

\newcommand{\triangleopen}{\raisebox{0pt}{\tikz{\draw[solid, line width = 1.0pt](4mm,0.15mm) -- (6mm,0.15mm) -- (5mm,1.95mm) -- cycle;
}}}

\newcommand{\downtriangleopen}{\raisebox{0pt}{\tikz{\draw[solid, line width = 1.0pt](4mm,1.95mm) -- (6mm,1.95mm) -- (5mm,0.15mm) -- cycle;
}}}

\newcommand{\diamondopen}{\raisebox{0pt}{
  \tikz{\draw[solid, line width = 1.0pt](-1mm,0mm) -- (0mm,1mm) -- (1mm,0mm) -- (0mm,-1mm) -- cycle;
  }}}


\newcommand{\rectanglesolidline}{\raisebox{0pt}{\tikz{\draw[solid,fill, line width = 1.0pt](4.mm,0)rectangle (5.8mm,1.8mm);
  \draw[-,solid,line width = 1.0pt](0.,1mm) -- (10.0mm,1mm)}}}

\newcommand{\rectangledashline}{\raisebox{0pt}{\tikz{\draw[solid,fill, line width = 1.0pt](4.mm,0)rectangle (5.8mm,1.8mm);
  \draw[-,dashed,line width = 1.0pt](0.,1mm) -- (10.0mm,1mm)}}}

\newcommand{\rectangledashdotline}{\raisebox{0pt}{\tikz{\draw[solid,fill, line width = 1.0pt](4.mm,0)rectangle (5.8mm,1.8mm);
  %\draw[-,dashdotted,line width = 1.0pt,dash pattern=on 1pt off 1.5pt on 2pt off 2pt](0.,1mm) -- (10.0mm,1mm)}}}
  \draw[-,dashdotted,line width = 1.0pt](0.,1mm) -- (10.0mm,1mm)}}}

\newcommand{\circlesolidline}{\raisebox{0pt}{\tikz{\draw[solid,fill, line width = 1.0pt](5mm,0) circle[radius=1.0mm];
  \draw[-,solid,line width = 1.0pt](0.,0.0mm) -- (10.0mm,0.0mm)}}}

\newcommand{\circledashline}{\raisebox{0pt}{\tikz{\draw[solid,fill, line width = 1.0pt](5mm,0) circle[radius=1.0mm];
  \draw[-,dashed,line width = 1.0pt](0.,1mm) -- (10.0mm,1mm)}}}

\newcommand{\circledashdotline}{\raisebox{0pt}{\tikz{\draw[solid,fill, line width = 1.0pt](5mm,0) circle[radius=1.0mm];
  %\draw[-,dashdotted,line width = 1.0pt,dash pattern=on 1pt off 1.5pt on 2pt off 2pt](0.,1mm) -- (10.0mm,1mm)}}}
  \draw[-,dashdotted,line width = 1.0pt](0.,1mm) -- (10.0mm,1mm)}}}

\newcommand{\trianglesolidline}{\raisebox{0pt}{\tikz{\draw[solid,fill, line width = 1.0pt](4mm,0.15mm) -- (6mm,0.15mm) -- (5mm,1.95mm) -- cycle;
  \draw[-,solid,line width = 1.0pt](0.,1.05mm) -- (10.0mm,1.05mm)}}}

\newcommand{\triangledashline}{\raisebox{0pt}{\tikz{\draw[solid,fill, line width = 1.0pt](4mm,0.15mm) -- (6mm,0.15mm) -- (5mm,1.95mm) -- cycle;
  \draw[-,dashed,line width = 1.0pt](0.,1.05mm) -- (10.0mm,1.05mm)}}}

\newcommand{\triangledashdotline}{\raisebox{0pt}{\tikz{\draw[solid,fill, line width = 1.0pt](4mm,0.15mm) -- (6mm,0.15mm) -- (5mm,1.95mm) -- cycle;
  %\draw[-,dashdotted,line width = 1.0pt,dash pattern=on 1pt off 1.5pt on 2pt off 2pt](0.,1.05mm) -- (10.0mm,1.05mm)}}}
  \draw[-,dashdotted,line width = 1.0pt](0.,1.05mm) -- (10.0mm,1.05mm)}}}

\newcommand{\downtrianglesolidline}{\raisebox{0pt}{\tikz{\draw[solid,fill, line width = 1.0pt](4mm,1.95mm) -- (6mm,1.95mm) -- (5mm,0.15mm) -- cycle;
  \draw[-,solid,line width = 1.0pt](0.,1.05mm) -- (10.0mm,1.05mm)}}}

\newcommand{\downtriangledashline}{\raisebox{0pt}{\tikz{\draw[solid,fill, line width = 1.0pt](4mm,1.95mm) -- (6mm,1.95mm) -- (5mm,0.15mm) -- cycle;
  \draw[-,dashed,line width = 1.0pt](0.,1.05mm) -- (10.0mm,1.05mm)}}}

\newcommand{\downtriangledashdotline}{\raisebox{0pt}{\tikz{\draw[solid,fill, line width = 1.0pt](4mm,1.95mm) -- (6mm,1.95mm) -- (5mm,0.15mm) -- cycle;
  %\draw[-,dashdotted,line width = 1.0pt,dash pattern=on 1pt off 1.5pt on 2pt off 2pt](0.,1.05mm) -- (10.0mm,1.05mm)}}}
  \draw[-,dashdotted,line width = 1.0pt](0.,1.05mm) -- (10.0mm,1.05mm)}}}

\newcommand{\diamondsolidline}{\raisebox{0pt}{
  \tikz{ \draw[solid,fill, line width = 1.0pt](4mm,0mm) -- (5mm,1mm) -- (6mm,0mm) -- (5mm,-1mm) -- cycle;
  \draw[-,solid,line width = 1.0pt](0.,0mm) -- (10.0mm,0mm)}}}

\newcommand{\diamonddashline}{\raisebox{0pt}{
  \tikz{ \draw[solid,fill, line width = 1.0pt](4mm,0mm) -- (5mm,1mm) -- (6mm,0mm) -- (5mm,-1mm) -- cycle;
  \draw[-,dashed,line width = 1.0pt](0.,0mm) -- (10.0mm,0mm)}}}

\newcommand{\diamonddashdotline}{\raisebox{0pt}{
  \tikz{ \draw[solid,fill, line width = 1.0pt](4mm,0mm) -- (5mm,1mm) -- (6mm,0mm) -- (5mm,-1mm) -- cycle;
  \draw[-,dashdotted,line width = 1.0pt](0.,0mm) -- (10.0mm,0mm)}}}

\DeclareRobustCommand{\legendsquaretest}[1]{%
    \tikz[baseline=(a.south)]{\node[#1, inner sep=.8ex, outer sep=0] (a) {}; \draw[->,solid,line width = 1.5pt](0,0) -- (5mm,0)(a) {}}%
}

