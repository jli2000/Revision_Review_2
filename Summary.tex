\section{Summary}
\label{sec:summary}

In this study, we have developed a \ac{DSRAR} framework for constructing surrogate models \blue{irrespective of the mutual dependence between the  components 
of random inputs} using limited training points. 
\blue{
To the best of our knowledge, this problem has not been
addressed by  previous \ac{UQ} studies based on polynomial 
chaos expansions. The \ac{DSRAR} framework does not assume  mutual 
independence between the components of random inputs and therefore can be
applied to \ac{UQ} in complex systems
where information about the underlying random distribution can be implicit.
}
%
%\blue{
%In particular, we do not assume that the random components are mutually independent. To the best 
%of our knowledge, the present framework takes a different starting point from the most of 
%the previous \ac{UQ} studies based on polynomial chaos expansion. Accordingly, the present method
%can be well-suited for uncertainty quantification of complex systems,
%where the knowledge of the underlying randomness can be implicit.
%}
%
To construct the surrogate model, this framework uses data-driven 
\blue{\ac{amdP}} basis 
construction and a sparsity-enhancing rotation procedure which leads 
to more accurate recovery of the sparse representation of the target function. 
%
%In particular, the present method does not assume that the random components
%are mutually independent. To the best of our knowledge, this is different from the majority of 
%the previous \ac{UQ} studies
%
%
%
%
The method benefits from both the orthonormal 
basis expansion and the enhanced sparsity of the expansion coefficients. 
%
\blue{With the assumption that there exists a sparse representation of the surrogate model},
the \ac{DSRAR} approach can be applied to challenging \ac{UQ} problems 
under two widely encountered situations: (\Rmnum{1}) probability measure 
implicitly represented by a large collection of samples and (\Rmnum{2}) 
non-Gaussian probability measures with explicit (analytical) forms.
%
For systems with explicit knowledge of the probability measure, our 
method exploits sparser representations of \acp{QoI} while retaining 
proper orthogonality with respect to rotated variables.
%
For systems with randomness implicitly represented by a large collection 
of random samples, we also proposed a heuristic method to construct a 
\emph{near-orthonormal} basis in addition to the exact orthonormal basis 
with respect to the discrete measure. The near-orthonormal basis 
shows {\color{blue}a} smaller basis bound and empirically yields more accurate 
representations. 
%
The numerical examples show the 
effectiveness of our method for realistic problems on \blue{quantifying 
uncertainty propagation in molecular system under 
conformational fluctuations} as well 
as \acp{PDE} with arbitrary underlying probability measures.

For future study, we note that several issues not considered in the 
present work could further improve the performance of the present \ac{DSRAR} framework.
The heuristic approach to constructing near-orthonormal basis introduced in 
this study yields smaller basis bounds and more accurate representations than 
existing methods. However, we do not have the theoretical analysis to formally 
show that the near-orthonormal basis is optimal and to establish the conditions 
under which it outperforms the exact orthonormal basis. It would be 
interesting to investigate 
different approaches of data-driven basis construction to further improve the 
properties of measurement matrix for \ac{CS} purposes. \textcolor{blue}{For instance,
if new data becomes available after the surrogate construction, it is worth
exploring how to use the new information to design more sophisticated 
(cross-validation) strategies to optimize the orthonormal threshold 
values and the basis construction procedure.} 
Furthermore, our study used a standard $\ell_1$ minimization approach 
for relaxing the \ac{CS} 
problem and recovering a sparse solution of the under-determined system.
However, other optimization approaches can be employed when the measurement 
matrix is highly coherent when $\ell_1$ minimization is not necessarily optimal.
Finally, it would be interesting to employ the developed \ac{DSRAR} approach for
\ac{UQ} study in other complex biological systems \cite{Bajaj_ACM_2016, Bajaj_JCB_2018}.
Such results will be presented in a future publication.
