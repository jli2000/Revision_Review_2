\section{Proof of Theorem~\ref{thm:exact_recovery}} \label{app:exact_recovery}

\begin{proof}
  Let $\bm v \in \rm{Ker}~\bm A$ and $\bm x \neq \bm c$ another solution of $\bm A \bm x = \bm b$. 
  To show that $\bm c$ is the unique $l_1$ minimizer of $\bm A \bm c  = \bm b$, it is sufficient if 
  \begin{equation}
    \Vert \bm v_{T_{\ba}}\Vert_1 < \Vert \bm v_{T_{\ba}^c}\Vert_1,
    \label{eq:null_condition}
  \end{equation}
  which gives
  \begin{equation}
    \begin{split}
      \Vert \bm c \Vert_1 &\le \Vert \bm c - \bm x_{T_{\ba}}\Vert_1 + \Vert  \bm x_{T_{\ba}}\Vert_1
      = \Vert \bm c_{T_{\ba}} - \bm x_{T_{\ba}}\Vert_1 + \Vert  \bm x_{T_{\ba}}\Vert_1
      = \Vert \bm v_{T_{\ba}}\Vert_1 + \Vert  \bm x_{T_{\ba}}\Vert_1  \\
      &< \Vert \bm v_{T_{\ba}^c}\Vert_1 + \Vert  \bm x_{T_{\ba}}\Vert_1 = \Vert \bm x \Vert_1.
    \end{split} 
  \end{equation}
  To satisfy \eqref{eq:null_condition}, we partition $T_{\ba}^c$ into $T_{\ba}^c =  T_{\ba,1}^c \bigcup  T_{\ba,2}^c \bigcup \cdots$, where $T_{\ba,1}^c$ is the index set of $s$ largest absolute entries of $\bm v$ in $T_{\ba}^c$, $T_{\ba,2 }^c$ is the index set of $s$ largest absolute entries of $\bm v$ in $T_{\ba}^c T_{\ba,1}^c$.
  Accordingly, 
  \begin{equation}
    \Vert \bm v_{T_{\ba}}\Vert_2^2 \le \frac{1}{1-\delta_s} \Vert \bm A \bm v_{T_{\ba}}\Vert_2 
    = \frac{1}{1-\delta_s} \sum_{k=1} \left\langle \bm A \bm v_{T_{\ba}}, \bm A (-\bm v_{T_{\ba,k}^c}) \right\rangle \le \frac{\theta_s}{1-\delta_s} \sum_{k=1} \Vert \bm v_{T_{\ba}}\Vert_2 \Vert \bm v_{T_{\ba,k}^c}\Vert_2,
  \end{equation}
  which gives $\Vert \bm v_{T_{\ba}}\Vert_2 \le \frac{\theta_s}{1-\delta_s} \sum_{k=1}  \Vert \bm v_{T_{\ba,k}^c}\Vert_2$.
  The remaining of the proof is straightforward and follows {\rm Theorem 2.6} of Rauhut~\cite{Rauhut_2010CsSM}.
  By the Cauchy-Schwarz inequality, we obtain 
  \begin{equation}
    \Vert \bm v_{T_{\ba}}\Vert_1 \le  \frac{\theta_s}{1-\delta_s} \left( \Vert \bm v_{T_{\ba}}\Vert_1 + \Vert \bm v_{T_{\ba}^c}\Vert_1 \right).
  \end{equation}
  Equation~\eqref{eq:null_condition} follows if $\frac{\theta_s}{1-\delta_s} < 0.5$. 
  %\left\langle %\bm\Psi \bm v_{T_{\ba}}, \bm\Psi (-\bm v T_{\ba,1}^c) + \bm\Psi (-\bm v T_{\ba,2}^c) + \cdots
  %\right\rangle
\end{proof}

\begin{rem}
  We emphasize that Theorem~\ref{thm:exact_recovery} holds only for the given index set $T_{\ba}$; it provides a metric to examine the recovery accuracy with respect to measurement matrix $\bm A$ and should not be viewed as the sufficient condition for exact recovery of \emph{arbitrary $s$-sparse vector} via $l_1$-minimization (see canonical references \cite{Candes_2005Decoding, Candes_2006Stablesrec,Rauhut_2010CsSM} for details). 
  Theorem~\ref{thm:exact_recovery} also indicates that, for the given index set $T_{\ba}$, small $\|\bm{A}^*_{T_{\ba}}\bm{A}_{T_{\ba}}-I\|_2$ will promote the recover of $\bm v_{T_{\ba}}$.
\end{rem}

\section{Measurement matrix and basis bounds}\label{app:basis_bound}
\subsection{Null space of measurement matrix from Section \ref{sec:basis_comparison}}
Let $\tilde{\bm c} = \bm c + \bm v$, $\bm v \in \textrm{Ker}~\bm{A}$ where $\bm{A}$ is the measurement matrix defined in \eqref{eq:matA}. 
From the null space property \cite{Rauhut_2010CsSM}, $\tilde{\bm c}$ does not fully recover $\bm c$ by 
$\ell_1$ minimization (i.e., equation \eqref{eq:ape_L1}) only if $\Vert  \tilde{\bm c} \Vert_1 < \Vert \bm c\Vert_1$. As a \emph{necessary 
condition} for the failure of recovery, it requires 
\begin{equation}
\left \Vert \bm v_{T_{\ba}}\right \Vert_1 > \left \Vert \bm v_{T_{\ba}^c}\right\Vert_1,
\label{eq:l1_failure}
\end{equation}  
where $T_{\ba}^c$ refers to the complement of $T_{\ba}$. Accordingly, different null space of measurement matrix $\bm{A}$ 
generally leads to different recovery error.

We examined the above necessary condition \eqref{eq:l1_failure} for different measurement matrices by randomly choosing a non-zero index set 
$T_{\ba}$ with $\left\vert T_{\ba}\right\vert = 50$ and $M = 180$.
For $\bm A$ constructed by both basis sets, we collected $1000$ normalized  $\bm v\in\rm{Ker}~\bm A $ that satisfy $\left \Vert \bm v_{T_{\ba}}\right \Vert_1 > \left \Vert \bm v_{T_{\ba}^c}\right\Vert_1$.
Figure~\ref{fig:contour_null_vector} shows the density contour of individual component $\left\vert \bm v_{i'}\right\vert$ in log-scale, where $i'$ refers to the index sorted by magnitude in descending order.
%%%
\begin{figure}[tbp]
  \center
  \subfigure[]{\includegraphics*[scale=0.25]{./Figure/Group_2/Logscale_null_vec_density_Ns_50_sam_180_orth}}
  \subfigure[]{\includegraphics*[scale=0.25]{./Figure/Group_2/Logscale_null_vec_density_Ns_50_sam_180_orth_bound}}
  \caption{{The null spaces of measurement matrices constructed by the exact and near-orthonormal bases are different under $\left \Vert \bm v_{T_{\ba}}\right \Vert_1 > \left \Vert \bm v_{T_{\ba}^c}\right\Vert_1$---a necessary
    condition for $\bm c$ not being recoverable exactly.}
  Density contour of the normalized null space vector component $\log\vert \rm {v}_{i'}\vert $ (sorted by magnitude) of the measurement matrix $\bm A$ constructed by orthogonal (a) and near-orthogonal basis functions (b) that satisfy $\left \Vert \bm v_{T_{\ba}}\right \Vert_1 > \left \Vert \bm v_{T_{\ba}^c}\right\Vert_1$ and $\Vert \bm v\Vert_2 = 1$.} \label{fig:contour_null_vector}
\end{figure}
%%%
The two basis sets demonstrate different distributions of $\log\left\vert \bm v_{i'}\right\vert$, which likely contribute 
to the different recovery errors shown in Figure~\ref{fig:err_no_sparse_vector}. 


\subsection{Basis bounds}
%It is useful to introduce the ``basis bound'' for the numerical results from our method.
The lower bound of the required number of samples $M$ given in Theorem \ \ref{thm:boundedBOS} suggests that bases with smaller basis bounds $K$ are preferred.
We expect that smaller basis bounds will correlate with higher accuracy representations.
For the constructed basis set $\psi_i(\bx), i = 1,\cdots, N$, we define the basis 
bound $\tilde{K}$ on the given data set $S$ by
\begin{equation}\label{eq:basis_bound}
\tilde{K} := \frac{1}{\vert S_{M_{\sigma}}\vert} \sum_{\bx\in S_{M_{\sigma}}}\vert k(\bm\xi)\vert,
\end{equation}
%\end{equation}
where the set $S_{M_{\sigma}}$ is defined by $S_{M_{\sigma}} = \left
\{\bx\Big\vert \vert k(\bx) - \mathbb{E}\left[k\right]\vert > M_{\sigma} \sigma\left[k\right], \bx \in S\right\}$. 
Here $\displaystyle k(\bx) :=\mathop{\max}_{i} \vert \psi_i(\bx) \vert$ denotes the maximum magnitude for 
an individual sampling point $\bx$, $\mathbb{E}\left[k\right]$ and $\sigma\left[k\right]$  represent the 
mean and the standard deviation of $k(\bx)$ on $S$ with respect to the discrete measure $\nu_S$.
%and $M_{\sigma}$ defines the spanned range for sampling $K$. 
In this study, we present $\tilde{K}$ as an indication of the difference between the exact and near-orthonormal basis function. 
In compressive sensing, the measurement matrix only consists of limited number of samples. Therefore, we 
employ the mean of the tails in the basis bounds as an indicator of the upper bound of the largest 
entry values from the measurement matrix. $M_{\sigma}$ defines the range of this tail set. 
We choose $M_{\sigma} = 5$ if not specified otherwise.

\begin{table}[!h]
\centering
\caption{$\tilde{K}$ of constructed basis set for Gaussian mixture system $d = 25$, $p = 2$ and $N_s = 1\times 10^5$.}
\begin{tabular}{C{8em}|C{6em} C{6em} C{6em} C{6em} C{6em}}
\hline\hline
$M_{\sigma}$ & $3$ & $4$ & $5$ & $6$ &$ \displaystyle \mathop{\max}_{\bx\in S} k(\bx)$ \\
\hline
$\tilde{K}_{\rm orth}$ & 10.359 & 12.048 & 13.895 & 15.513 & 22.208\\ 
$\tilde{K}_{\rm near-orth}$ & 9.622 & 11.196 & 12.867 & 14.448 & 18.790\\ 
\hline\hline
\end{tabular}
\label{tab:GM_d_25_p_2}
\end{table}

Following the definition by Equation \eqref{eq:basis_bound}, we examine the basis bound $\tilde{K}$ of the numerical
examples presented in this study. Table \ref{tab:GM_d_25_p_2} shows the results of Gaussian mixture system  
$\left\{\bx^{(i)}\right\}, i = 1, \cdots, N_s$ with $N_s = 1\times 10^5$, $d = 25$ and $p = 2$ which is defined in Section \ref{sec:basis_comparison}. For different
values of $M_{\sigma}$, $\tilde{K}$ of the near orthogonal basis shows consistently smaller values than
the values of the exact orthogonal basis set. 

Table \ref{tab:GM_d_25_p_3} shows the basis bound $\tilde{K}$ of the Gaussian mixture system which is studied 
in Section \ref{sec:high_d_poly} with $N_s = 2\times 10^5$, 
$d = 25$ and $p = 3$ . The values of $\tilde{K}$ for the near orthogonal basis are consistently smaller than 
the value for the exact orthogonal basis set no matter on the original random sample set or the rotated sample set. 
Furthermore, we present the basis bounds on the rotated sampling set $\left\{\bm\chi_M^{\left(i\right)}\right\}_{i=1}^{N_s}$, where
the subscript ``$M$'' refers to the different number of training points utilized to construct the surrogate model
$X(\bx)$. The near-orthogonal basis yields smaller $\tilde{K}$ than the exact orthogonal basis in each case. 

Similarly, Table \ref{tab:BioMol_d_12_p_4} shows $\tilde{K}$ 
of the constructed basis for uncertainty quantification of the molecular solvation energy 
($d = 12$, $p=4$ and $N_s = 2\times 10^5$), which is studied in Section \ref{sec:mole_example}. The near-orthogonal basis yields smaller values consistently for different number ($\bm{\chi}_M$) of training points. 

\begin{table}[!h]
\centering
\caption{$\tilde{K}$ of constructed basis set for Gaussian mixture system $d = 25$, $p = 3$ and $N_s = 2\times 10^5$.}
\begin{tabular}{C{8em}|C{6em} C{6em} C{6em} C{6em} C{6em}}
\hline\hline
 & $\bx$ & $\bm{\chi}_{M=400}$ & $\bm{\chi}_{M=1200}$ & $\bm{\chi}_{M=1600}$ & $\bm{\chi}_{M=2400}$ \\
\hline
$\tilde{K}_{\rm orth}$ & 32.497 & 32.522 & 32.079 & 33.142 & 32.308 \\ 
$\tilde{K}_{\rm near-orth}$ & 28.320 & 29.811 & 29.407 & 29.512 & 29.192\\ 
\hline\hline
\end{tabular}
\label{tab:GM_d_25_p_3}
\end{table}

\begin{table}[!h]
\centering
\caption{$\tilde{K}$ of constructed basis set for molecular system $d = 12$, $p = 4$ and $N_s = 2\times 10^5$.}
\begin{tabular}{C{8em}|C{6em} C{6em} C{6em} C{6em} C{6em}}
\hline\hline
 & $\bm{\chi}_{M=80}$ & $\bm{\chi}_{M=160}$ & $\bm{\chi}_{M=240}$ & $\bm{\chi}_{M=320}$ & $\bm{\chi}_{M=400}$  \\
\hline
$\tilde{K}_{\rm orth}$ & 40.596 & 39.914 & 39.789 & 39.218 & 39.142 \\ 
$\tilde{K}_{\rm near-orth}$ & 39.970 & 39.278 & 39.290 & 38.528 & 38.631\\ 
\hline\hline
\end{tabular}
\label{tab:BioMol_d_12_p_4}
\end{table}

\section{Other metrics for the surrogate model}
\blue{
Besides the relative $l_2$ error, we have also computed the predictivity coefficients $Q_2$ for 
the test cases of Gaussian mixture (with $d = 25$ and
$p = 3$) and molecular systems. Similar to Marrel et al.~\cite{Marrel_Q2_2009}, $Q_2$ is defined by
\begin{equation}
Q_2 = 1 - \int(f(\bx) - \tilde{f}(\bx))^2 \dif \nu_{S_2}(\bx) \big/ \int \left(f(\bx) - \bar{f}\right)^2 \dif \nu_{S_2}(\bx) ,
\end{equation}
where $\bar{f}$ represents the mean of QoI on $S_2$. The results are shown in Tab.~\ref{tab:Q_2_GM_mol}, where the surrogate models are
constructed by the our data-driven basis approach. }

\blue{
\begin{table}[!h]
\centering
\caption{The predictivity coefficient $Q_2$ for polynomial function with 
  Gaussian Mixture measure ($d = 25$ and $p = 3$) and the molecular system for
    solvation energy and SASA of atom $\rm{H9}$.}
\begin{tabular}{C{8em} C{3em} C{4em} C{4em} C{4em} C{4em} C{4em}}
\hline\hline
molecule solvation & $M$ & 80 & 160 &  240 & 320 & 400 \\
 & $Q_2$ & 0.995715 & 0.999132 &  0.999731 & 0.999864 & 0.999911 \\
\hline
molecule SASA & $M$ & 200 & 300 &  400 & 500 & 600 \\
 & $Q_2$ & 0.988675 & 0.996069 &  0.998272 & 0.998709 & 0.999027 \\
\hline
Gaussian Mixture & $M$ & 200 & 300 &  400 & 500 & 600 \\
 & $Q_2$ & 0.998372 & 0.999347 &  0.999844 & 0.999892 & 0.999941 \\
\hline\hline
\end{tabular}
\label{tab:Q_2_GM_mol}
\end{table}
}
\blue{
With the constructed surrogate model, we can compute the \textcolor{red}{Sobol'} sensitivity indices for 
\ac{QoI} with dependent random variables. In brief, $f(\bx)$ is expanded by
\begin{equation}
f(\bx) = \eta_0(\bx) + \sum_{\bm\beta \in \Theta^d} \eta_{\bm\beta}(\bx),
\end{equation}
where $\Theta^d$ represents the collection of all subsets of $[1~:~d]$ and $\eta_{\bm\beta}(\bx)$
satisfies 
%\begin{equation}
$\mathbb{E}\left[\eta_{\bm\alpha}, \eta_{\bm\beta}\right] = 0$, if $\bm\alpha \subset \bm\beta$.
The sensitivity index $S_{\bm\beta}$ is given by
\begin{equation}
S_{\bm\beta} = \frac{\mathbb{V}(\eta_{\bm\beta}) + \sum_{\ba \cap \bm\beta \neq \ba, \bm\beta} \rm{Cov}(\eta_{\ba}, \beta_{\bm\beta})}
    {\mathbb{V}(f)}
\end{equation}
where $\mathbb{V}(\cdot)$ refers to the variance on $\nu_S$. 
We refer to Chastaing et al.~\cite{Chastaing_sobol_2014} for the details. Fig.~\ref{tab:Q_2_GM_mol} shows the first-order sensitivity indices for the test cases of Gaussian mixture systems 
($d = 25$, $p = 3$) and the molecular systems, where the surrogate models are
constructed by the data-driven basis approach using $M = 800$, $M = 240$ and $M = 600$ 
training points, respectively. The dominant components are on the 
dimensions $(1, 2, 3, 6, 11, 13, 14, 15, 16, 20, 22, 24, 25)$, $(1, 2, 5)$ and $(1, 2, 4, 5, 7)$ ($90\%$ of 
total variance).
}
%%%
\begin{figure}[tbp]
  \center
  \subfigure[]{\includegraphics*[scale=0.21]{./Figure/Group_9/Sobol_indices_rand_poly_d_25_p_3}}
  \subfigure[]{\includegraphics*[scale=0.21]{./Figure/Group_9/Sobol_indices_mol_blb}}
  \caption{The first-order \textcolor{red}{Sobol'} sensitivity indices for (a) polynomial function with 
  Gaussian mixture measure ($d = 25$, $p = 3$) (b) molecular system for
    solvation energy (``\textcolor{red}{\protect\rectanglesolidline}'')
    and \ac{SASA} of atom $\rm{H9}$(``\textcolor{blue}{\protect\diamonddashdotline}'').
   } \label{fig:sobol_GM_mol}
\end{figure}
%%%

\section{Generation of the Gaussian mixture data set}
\label{app:GM_generation}
\blue{
We used Matlab to generate the Gaussian mixture data set in Sec.~\ref{sec:basis_comparison}
by calling the function \\ \texttt{gmdistribution(}$\bm\mu$, $\left\{\Sigma_i\right\}_{i=1}^3$, $\bm a$\texttt{)} with 
$\bm a = (0.5358, 0.1281, 0.3361)$. $\bm\mu$ is a $25\times 3$ random 
matrix with i.i.d.\ entries on $U[-2.5, 2.5]$. $\left\{\Sigma_i\right\}_{i=1}^3$ is a 
$25\times25\times3$ array where $\Sigma_i$ is defined by
\begin{equation}
  \bm\Sigma_i = (\bm \Upsilon_i {\bm \Upsilon_i}^T + \mathbf{I})/4,
\end{equation}
where $\bm\Upsilon_i$ is a random matrix with i.i.d.\ entries from $\mathcal{U}[0,1]$ for $i=1,2,3$.
$\bm\mu$ and $\Upsilon_i$ are generated by calling the Matlab function \texttt{rand()} with
random number seed $200$.
}


\section{Molecular Dynamics simulation and calculation details}
\label{app:sim}
We performed all-atom \ac{MD} simulation of benzyl bromide in water using GROMACS 5.1.2 \cite{GROMACS}.
The simulation system included a benzyl bromide molecule (see Figure \ref{fig:mol_blb} for the molecular structure) 
 and 1011 water molecules.
The \ac{GAFF} \cite{RN1} was used for the benzyl bromide parmameters.
The partial charges of benzyl bromide molecule were calculated by RESP method \cite{RN2}.
Bond lengths of benzyl bromide were constrained using the LINCS algorithm \cite{RN3}.
The water molecule was modeled with the rigid TIP3P water model \cite{RN4}.
The bond lengths and angles were held constant through the SETTLE algorithm \cite{RN5}.
The system was equilibrated in the isothermal-isobaric ensemble for 10 ns at 300K and 1 bar after energy minimization.
The van der Waals cut-off radii was 1.0 nm.
Long-range electrostatics were calculated using a Particle Mesh Ewald (PME) summation with grid spacing of 0.12 nm.
The time step was 2 fs.
Isobaric-isothermal simulations were equilibrated using a V-rescale thermostat and Berendsen barostat.
Following equilibration, the simulation was run for a production period of 20 $\mu$s in a NVT ensemble with a Nos\'e-Hoover thermostat.
The trajectory was stored every 10000 time steps.

\begin{figure}[htbp]
\center
\includegraphics*[scale=0.6]{./Figure/Group_8/blb_structure}
\caption{Sketch of the molecule benzyl bromide with labeled atoms.}
\label{fig:mol_blb}
\end{figure}


APBS calculations \cite{BakerSJHM01, APBS_2018} were performed with $129^3$ grid points over a $40 \times 40 \times 40$ \AA$^3$ coarse grid domain with focusing to a $14 \times 14 \times 14$ \AA$^3$ fine grid domain with the grid origin located at the geometric center of the molecule.
The Poisson equation was solved with Dirichlet boundary conditions based on the asymptotic behavior of multiple point charges in a homogeneous dielectric medium.
The dielectric coefficient inside the domain used a van der Waals molecular volume definition with a dielectric value of 2.0 inside the molecule and 78.0 outside the molecule.
Charges were modeled by Dirac delta functions but discretized to the finite difference grid points using a cubic spline approximation.

